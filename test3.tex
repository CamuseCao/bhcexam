\documentclass[answers]{BHCexam}
\begin{document}

\maketitle
\mininotice

\begin{questions}
\tiankong
\question 已知~$\bm{a}=(k,-9)$、$\bm{b}=(-1,k)$, $\bm{a}$~与~$\bm{b}$~为平行向量,
    则~$k=$\sixb.
\begin{solution}
$\pm3$
\end{solution}
\question 若函数~$f(x)=x^{6m^2-5m-4}\,(m\in\mathbb{Z})$~的图像关于~$y$~轴对称,
    且~$f(2)<f(6)$, 则~$f(x)$~的解析式为\tenb.
\begin{solution}
$f(x)=x^{-4}$
\end{solution}

\question 若~$f(x+1)=x^2\,(x\leq0)$, 则~$f^{-1}(1)=$\sixb.
\begin{solution}
0
\end{solution}

\question 在~$b\g$~糖水中含糖~$a\g$\,($b>a>0$), 若再添加~$m\g$~糖~($m>0$),
    则糖水就变甜了.试根据这个事实, 提炼一个不等式\tenb.
\begin{solution}
$\dfrac{a+m}{b+n}>\dfrac{a}{b}$
\end{solution}

\question 已知~$f(x)=1-\rc_8^1x+\rc_8^2x^2-\rc_8^3x^3+\cdots+\rc_8^8x^8$,
    则~$f\big(\dfrac{1}{2}+\dfrac{\sqrt{3}}{2}\ri\big)$~的值是\sixb\twob{}.
\begin{solution}
$-\dfrac{1}{2}-\dfrac{\sqrt{3}}{2}\ri$
\end{solution}

\question 自然数~$1,2,3,\ldots,10$~的方差记为~$\sigma^2$,
    其中的偶数~$2,4,6,8,10$~的方差记为~$\sigma_1^2$,
    则~$\sigma^2$~与~$\sigma_1^2$~的大小关系为~$\sigma^2$\sixb$\sigma_1^2$.
\begin{solution}
$>$
\end{solution}

\question 若~$\theta$~为三角形的一个内角, 且~$\sin\theta+\cos\theta=\dfrac{2}{3}$,
    则方程~$x^2\csc\theta-y^2\sec\theta=1$~表示的曲线的焦点坐标是\sixb{}.
\begin{solution}
$\big(\pm\dfrac{\sqrt{6}}{3},0\big)$
\end{solution}

\question 高为~$h$~的棱锥被平行于棱锥底面的截得棱台侧面积是
    原棱锥的侧面积的~$\dfrac{5}{9}$,
    则截得的棱台的体积与原棱锥的体积之比是\sixb.
\begin{solution}
$19:27$
\end{solution}

\question 以椭圆~$\dfrac{x^2}{169}+\dfrac{y^2}{144}=1$~的右焦点为圆心,
    且与双曲线~$\dfrac{x^2}{9}-\dfrac{y^2}{16}=1$~的渐近线相切的圆方程是\tenb.
\begin{solution}
$(x-5)^2+y^2=16$
\end{solution}

\question 若~$\sqrt{\,\sin x}$~是有理数且~$x$~不是~$\dfrac{\pi}{6}$~的整数倍,
    则~$x$~可能取的值是\tenb.(只要求写出一个)
\begin{solution}
$\arcsin\dfrac{1}{4}$ 等
\end{solution}

\question 马路上有编号~1~到~10~的~10~盏路灯, 为节约用电又不影响照明,
    可以关掉其中的~3~盏, 但又不能同时关掉相邻的两盏, 也不能关掉两端的路灯,
    满足条件的关灯方法有\sixb{}种.
\begin{solution}
20
\end{solution}
\question 以椭圆~$\dfrac{x^2}{169}+\dfrac{y^2}{144}=1$~的右焦点为圆心,
    且与双曲线~$\dfrac{x^2}{9}-\dfrac{y^2}{16}=1$~的渐近线相切的圆方程是\tenb.
\begin{solution}
$(x-5)^2+y^2=16$
\end{solution}

\question 若~$\sqrt{\,\sin x}$~是有理数且~$x$~不是~$\dfrac{\pi}{6}$~的整数倍,
    则~$x$~可能取的值是\tenb.(只要求写出一个)
\begin{solution}
$\arcsin\dfrac{1}{4}$ 等
\end{solution}

\question 马路上有编号~1~到~10~的~10~盏路灯, 为节约用电又不影响照明,
    可以关掉其中的~3~盏, 但又不能同时关掉相邻的两盏, 也不能关掉两端的路灯,
    满足条件的关灯方法有\sixb{}种.
\begin{solution}
20
\end{solution}
\xuanze
\question 已知集合~$A=\left\{\,x\mid \abs{x-1}<3\,\right\}$,
集合~$B=\{\,y\mid y=x^2+2x+1,x\in\mathbb{R}\,\}$, 则~$A\cap
\complement_U B$~为
\begin{choices}
\choice $[\,0,4)$
\choice $(-\infty,-2\,]\cup[4,+\infty)$
\choice $(-2,0)$
\choice $(0,4)$
\end{choices}
\begin{solution}
C
\end{solution}

\question 若~$a$、$b$~是直线, $\alpha$、$\beta$~是平面,
则以下命题中真命题是 
\begin{choices}
\choice 若~$a$、$b$~异面, $a\subset\alpha$,
$b\subset\beta$, 且~$a\perp b$, 则~$\alpha\perp\beta$
\choice 若~$a\pingxing b$, $a\subset\alpha$, $b\subset\beta$,
则~$\alpha\pingxing\beta$
\choice 若~$a\pingxing \alpha$,
$b\subset\beta$, 则~$a$、$b$ 异面
\choice 若~$a\perp b$, $a\perp\alpha$,
$b\perp\beta$, 则~$\alpha\perp\beta$
\end{choices}
\begin{solution}
D
\end{solution}

\question 已知集合~$A=\left\{\,x\mid \abs{x-1}<3\,\right\}$,
集合~$B=\{\,y\mid y=x^2+2x+1,x\in\mathbb{R}\,\}$, 则~$A\cap
\complement_U B$~为
\begin{choices}
\choice $[\,0,4)$
\choice $(-\infty,-2\,]\cup[4,+\infty)$
\choice $(-2,0)$
\choice $(0,4)$
\end{choices}
\begin{solution}
C
\end{solution}

\question 若~$a$、$b$~是直线, $\alpha$、$\beta$~是平面,
则以下命题中真命题是 
\begin{choices}
\choice 若~$a$、$b$~异面, $a\subset\alpha$,
$b\subset\beta$, 且~$a\perp b$, 则~$\alpha\perp\beta$
\choice 若~$a\pingxing b$, $a\subset\alpha$, $b\subset\beta$,
则~$\alpha\pingxing\beta$
\choice 若~$a\pingxing \alpha$,
$b\subset\beta$, 则~$a$、$b$ 异面
\choice 若~$a\perp b$, $a\perp\alpha$,
$b\perp\beta$, 则~$\alpha\perp\beta$
\end{choices}
\begin{solution}
D
\end{solution}

\jianda
\question 已知复数~$z$ 满足:$\abs{z}-z^*=\dfrac{10}{1-w\ri}$(其中~$z^*$
是~$z$ 的共轭复数).
\begin{parts}
\part[7] 求复数~$z$;
\part[7] 若复数~$w=\cos\theta+\ri\sin\theta\,(\theta\in\mathbb{R})$, 求~$\abs{z-2}$ 的取值范围.
\end{parts}

\begin{solution}
\begin{parts}
\part $z=3+4\ri$
\part $\abs{z-w}\in[4,6]$
\end{parts}
\end{solution}

\question[14] 函数~$f(x)=4\sin\dfrac{\pi}{12}x\cdot\sin
    \left(\dfrac{\pi}{2}+\dfrac{\pi}{12}x\right),x\in[a,a+1]$,
    其中常数~$a\in[0,5]$, 求函数~$f(x)$ 的最大值~$g(a)$.

\begin{solution}
略
\end{solution}

\question[16] 函数~$f(x)=4\sin\dfrac{\pi}{12}x\cdot\sin
    \left(\dfrac{\pi}{2}+\dfrac{\pi}{12}x\right),x\in[a,a+1]$,
    其中常数~$a\in[0,5]$, 求函数~$f(x)$ 的最大值~$g(a)$.

\begin{solution}
略
\end{solution}

\question 已知复数~$z$ 满足:$\abs{z}-z^*=\dfrac{10}{1-w\ri}$(其中~$z^*$
是~$z$ 的共轭复数).
\begin{parts}
\part[8] 求复数~$z$;
\part[8] 若复数~$w=\cos\theta+\ri\sin\theta\,(\theta\in\mathbb{R})$, 求~$\abs{z-2}$ 的取值范围.
\end{parts}

\begin{solution}
\begin{parts}
\part $z=3+4\ri$
\part $\abs{z-w}\in[4,6]$
\end{parts}
\end{solution}


\question[18] 函数~$f(x)=4\sin\dfrac{\pi}{12}x\cdot\sin
    \left(\dfrac{\pi}{2}+\dfrac{\pi}{12}x\right),x\in[a,a+1]$,
    其中常数~$a\in[0,5]$, 求函数~$f(x)$ 的最大值~$g(a)$.

\begin{solution}
略
\end{solution}

\end{questions}
\end{document}
