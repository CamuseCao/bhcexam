\documentclass[a4paper, adobe]{BHCexam}
\pagestyle{fancy}
\fancyfoot[C]{\kaishu \small 第 \thepage 页 共 \pageref{lastpage} 页}
\begin{document}
\write18{wget 'http://www.mathcrowd.cn/index.php?r=worksheet/qrcode&id=E8zR' -O qrcode.png}
\logo{qrcode.png}
\title{2020_01_29 导数单调性}
\subtitle{}
\notice{微信关注公众号: \textbf{橘子数学}}
\author{}
\date{}
\maketitle
\begin{groups}
\group{填空}{}
\begin{questions}[s]
\begin{minipage}{\textwidth}
\question[0] 已知函数$f(x)=2x ^{3} -ax ^{2} +b$.
\begin{subquestions}
    \subquestion 讨论$f(x)$的单调性; 
    \subquestion 是否存在$a$,$b$,使得$f(x)$在区间$[0 , 1]$的最小为$-1$且最大值为$1$?若存在,求出$a$,$b$的所有值;若不存在,说明理由.
\end{subquestions}
\begin{solution}{8cm}

\end{solution}
\end{minipage}
\begin{minipage}{\textwidth}
\question[0] 已知函数 $y=a\left(x^3-3x\right)\left(a\neq 0\right)$ 在区间 $\left(-1,1\right)$ 上单调递增,则 a 的取值范围是\key{$\left(-\infty,0\right)$}.
\begin{solution}{4cm}

\end{solution}
\end{minipage}
\begin{minipage}{\textwidth}
\question[0] 函数 $y=x^2\left(x+3\right)$ 的单调减区间是\key{$\left(-2,0\right)$}.
\begin{solution}{4cm}

\end{solution}
\end{minipage}
\begin{minipage}{\textwidth}
\question[0] 函数 $f\left(x\right) = \dfrac{1}{2}{x^2} + x\ln x - 2x$ 的单调递减区间为\key{$\left(0,1\right)$}.
\begin{solution}{4cm}

\end{solution}
\end{minipage}
\begin{minipage}{\textwidth}
\question[0] 函数 $f\left(x\right)=\left(4-x\right){\mathrm {e}}^x$ 的单调递减区间是 \key{D}..
\fourchoices{$\left(-\infty ,4\right)$}{$\left(-\infty ,3\right)$}{$\left(4,+\infty \right)$}{$\left(3,+\infty \right)$}
\begin{solution}{4cm}

\end{solution}
\end{minipage}
\begin{minipage}{\textwidth}
\question[0] 设函数 $f\left(x\right)=x\left({\mathrm {e}}^x-1\right)-ax^2$.
\begin{subquestions}
    \subquestion 若 $a=\dfrac12$,求 $f\left(x\right)$ 的单调区间;
    \subquestion 若当 $x\geqslant0$ 时 $f\left(x\right)\geqslant0$,求 a 的取值范围.
\end{subquestions}
\begin{solution}{8cm}

\end{solution}
\end{minipage}
\end{questions}
\group{选择}{}
\begin{questions}[s]
\end{questions}
\group{解答}{}
\begin{questions}[s]
\end{questions}
\end{groups}
\label{lastpage}
\end{document}