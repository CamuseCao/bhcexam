% \iffalse meta-comment
%
% Copyright (C) 2011 by Charles Bao <charley792@gmail.com>
%
% This file is part of the BHCexam package project.
% ---------------------------------------------------
%
% It may be distributed under the conditions of the LaTeX Project Public
% License, either version 1.2 of this license or (at your option) any
% later version. The latest version of this license is in
%    http://www.latex-project.org/lppl.txt
% and version 1.2 or later is part of all distributions of LaTeX
% version 1999/12/01 or later.
%
%<*!(cfg|fd)>
% \fi
%
%% \CharacterTable
%%  {Upper-case    \A\B\C\D\E\F\G\H\I\J\K\L\M\N\O\P\Q\R\S\T\U\V\W\X\Y\Z
%%   Lower-case    \a\b\c\d\e\f\g\h\i\j\k\l\m\n\o\p\q\r\s\t\u\v\w\x\y\z
%%   Digits        \0\1\2\3\4\5\6\7\8\9
%%   Exclamation   \!     Double quote  \"     Hash (number) \#
%%   Dollar        \$     Percent       \%     Ampersand     \&
%%   Acute accent  \'     Left paren    \(     Right paren   \)
%%   Asterisk      \*     Plus          \+     Comma         \,
%%   Minus         \-     Point         \.     Solidus       \/
%%   Colon         \:     Semicolon     \;     Less than     \<
%%   Equals        \=     Greater than  \>     Question mark \?
%%   Commercial at \@     Left bracket  \[     Backslash     \\
%%   Right bracket \]     Circumflex    \^     Underscore    \_
%%   Grave accent  \`     Left brace    \{     Vertical bar  \|
%%   Right brace   \}     Tilde         \~}
%%
%
% \CheckSum{0}
%
% \iffalse meta-comment
%</!(cfg|fd)>
%
%<*driver>
\ProvidesFile{BHCexam.dtx}
%</driver>
%
%<cls>\NeedsTeXFormat{LaTeX2e}[1995/12/01]
%<cls>\ProvidesClass{BHCexam}
%<cfg>\ProvidesFile{BHCexam.cfg}
  [2011/07/16 v0.0 BHCexam
%<cls>   document class]
%<cfg>   configuration file]
%
%<*driver>
   bundle source file]
%</driver>
%
%<*driver>
\documentclass{ltxdoc}
\usepackage{ctex}
\usepackage{hyperref}
 \topmargin 1 true cm
 \oddsidemargin 2.5 true cm
 \evensidemargin 2.5 true cm
 \textheight 21 true cm
 \textwidth 14 true cm
\EnableCrossrefs
 %\DisableCrossrefs % Say \DisableCrossrefs if index is ready
\CodelineIndex
\RecordChanges      % Gather update information
 %\OnlyDescription  % comment out for implementation details
 %\OldMakeindex     % use if your MakeIndex is pre-v2.9
\hypersetup{colorlinks,linkcolor=blue,citecolor=blue}
\begin{document}
  \DocInput{BHCexam.dtx}
\end{document}
%</driver>
%
% \fi
%
%
% \changes{v0.0}{2011/07/16}{Initial version}
%
%
% \DoNotIndex{\begin,\end,\begingroup,\endgroup}
% \DoNotIndex{\ifx,\ifdim,\ifnum,\ifcase,\else,\or,\fi}
% \DoNotIndex{\let,\def,\xdef,\newcommand,\renewcommand}
% \DoNotIndex{\expandafter,\csname,\endcsname,\relax,\protect}
% \DoNotIndex{\Huge,\huge,\LARGE,\Large,\large,\normalsize}
% \DoNotIndex{\small,\footnotesize,\scriptsize,\tiny}
% \DoNotIndex{\normalfont,\bfseries,\slshape,\interlinepenalty}
% \DoNotIndex{\hfil,\par,\vskip,\vspace,\quad}
% \DoNotIndex{\centering,\raggedright}
% \DoNotIndex{\c@secnumdepth,\@startsection,\@setfontsize}
% \DoNotIndex{\ ,\@plus,\@minus,\p@,\z@,\@m,\@M,\@ne,\m@ne}
% \DoNotIndex{\@@par}
%
%
% \GetFileInfo{BHCexam.dtx}
%
%
% \MakeShortVerb{\|}
% \setcounter{StandardModuleDepth}{1}
%
%
% \newcommand{\ctex}{\texttt{ctex}}
% \newcommand{\BHCexam}{\texttt{BHCexam}}
%
%
% \setlength{\parskip}{0.75ex plus .2ex minus .5ex}
% \renewcommand{\baselinestretch}{1.2}
%
%
% \makeatletter
% \def\parg#1{\mbox{$\langle${\it #1\/}$\rangle$}}
% \def\@smarg#1{{\tt\string{}\parg{#1}{\tt\string}}}
% \def\@marg#1{{\tt\string{}{\rm #1}{\tt\string}}}
% \def\marg{\@ifstar\@smarg\@marg}
% \def\@soarg#1{{\tt[}\parg{#1}{\tt]}}
% \def\@oarg#1{{\tt[}{\rm #1}{\tt]}}
% \def\oarg{\@ifstar\@soarg\@oarg}
% \makeatother
%
%
% \title{\bf \BHCexam~宏包说明\thanks
%   {本文件版本号为~\fileversion{},最后修改日期~\filedate{}。}}
% \author{\it 鲍宏昌\thanks{charley792@gmail.com}}
% \date{\small 打印日期:~\today}
% \maketitle
%
%
% \begin{abstract}
% \BHCexam~宏包提供了一个中学教学设计的~\LaTeX{}~文档类,
%
% \BHCexam~主要文件包括~\texttt{BHCexam.cls}~文档类和配置文件~\texttt{BHCexam.cfg}。
%
% \BHCexam~宏包由~鲍宏昌~制作并负责维护。
% \end{abstract}
%
%
% \tableofcontents
%
%
% \section{简介}
%
% TODO
%
%
% \section{使用帮助}
%
% TODO
%
%
% \subsection{选项}
%
% TODO
%
%
% \subsection{基本命令}
%
% TODO
%
%
%
%
% \StopEventually{
% } ^^A end StopEventually
%
% \section{源代码说明}
%
% \subsection{选项}
%
% 处理~\BHCexam~文档类的选项
%
% \begin{macro}{[(no)marginl]}
% 是否有装订线
%    \begin{macrocode}
%<*cls>
\newif\if@marginline \@marginlinefalse
\DeclareOption{marginl}{\@marginlinetrue}
\DeclareOption{nomarginl}{\@marginlinefalse}
%</cls>
%    \end{macrocode}
% \end{macro}
% 
% \begin{macro}{[16kpapaper]}
% 16k纸张大小设置,默认选项为a4paper
%    \begin{macrocode}
%<*cls>
\DeclareOption{16kpaper}
 {\setlength\paperheight {260mm}
  \setlength\paperwidth  {184mm}}
%</cls>
%    \end{macrocode}
% \end{macro}
%
% 把没有定义的选项传递给底层的文档类,并加载缺省的文档类
%    \begin{macrocode}
%<cls>\DeclareOption*{\PassOptionsToClass{\CurrentOption}{exam}}
%<cls>\ProcessOptions
%<cls>\LoadClass[addpoints]{exam}
%    \end{macrocode}
%
% \subsection{宏包}
%
% 装入常用的宏包
%    \begin{macrocode}
%<*cls>
\RequirePackage{amsmath,amsthm,amsfonts,amssymb,bm}
\RequirePackage{graphicx}
\RequirePackage{picins}
\RequirePackage{ifthen}
\RequirePackage{ctex}
%</cls>
%    \end{macrocode}
%
% \subsection{外观}
%
% 页眉、页脚设置
%    \begin{macrocode}
%<*cls>
\extraheadheight{-0.5in}
\pagestyle{headandfoot}
%\runningheadrule
%\firstpageheadrule
%\runningfootrule
%\firstpagefootrule
\header{}{}{}
\footer{}{\small \kaishu{\@kemu}\quad 第~\thepage~页(共~\numpages~页)}{}
%</cls>
%    \end{macrocode}
%
% 装订线
%    \begin{macrocode}
%<*cls>
\if@marginline
\extrawidth{-0.5in}
\newsavebox{\zdx}
\sbox{\zdx}{
  \begin{minipage}{412pt}
    \kaishu \centering
    班级\underline{\hspace{3cm}}学号\underline{\hspace{3cm}}姓名\underline{\hspace{3cm}}

    \vspace{3mm}
    \dotfill{}装\dotfill{}订\dotfill{}线\dotfill{}
  \end{minipage}
}
%\reversemarginpar
%\marginpar{
%  \begin{minipage}{10mm}
%    \vspace{5cm} 
%    \rotatebox[origin=c]{90}{\usebox{\zdx}}
%  \end{minipage}
%}
\fi
%</cls>
%    \end{macrocode}
%
% \begin{macro}{\biaoti}
% \begin{macro}{\kemu}
% 标题
%    \begin{macrocode}
%<*cfg>
\def\@biaoti{2011年全国普通高等学校招生统一考试(上海卷)}
\def\@kemu{数学(理科)}
%</cfg>
%<*cls>
\newcommand\biaoti[1]{\def\@biaoti{#1}}
\newcommand\kemu[1]{\def\@kemu{#1}}
\renewcommand\maketitle{
  \begin{center}{\heiti \Large{\@biaoti}}\end{center}
  \begin{center}{\heiti \Large{\@kemu}}\end{center}
}
%</cls>
%    \end{macrocode}
% \end{macro}
% \end{macro}
% 
% \begin{macro}{\xuzhi}
% \begin{macro}{\duanxz}
% \begin{macro}{\shexuzhi}
% 考试须知
%    \begin{macrocode}
%<*cfg>
\def\@shijian{120}
\def\@zongfen{150}
\def\@tishu{23}
%</cfg>
%<*cls>
\newcommand\shexuzhi[3]{
  \def\@shijian{#1}
  \def\@zongfen{#2}
  \def\@tishu{#3}
}
\newcommand\duanxz{
  \begin{center}{
    \kaishu (本试卷满分~\@zongfen~分, 考试时间~\@shijian~分钟)}
  \end{center}
}
\newcommand\xuzhi{
  \begin{flushleft}
    \heiti 
    考生注意:
    \begin{enumerate}
      \item 答卷前,考生务必在答题纸上将班级、姓名、学号填写清楚.
      \item 本试卷共有~\@tishu~道试题,满分~\@zongfen~分,考试~\@shijian~分钟.
    \end{enumerate}
  \end{flushleft}
}
%</cls>
%    \end{macrocode}
% \end{macro}
% \end{macro}
% \end{macro}
%
% \subsection{格式}
%
% 缩进
%    \begin{macrocode}
%<*cls>
\renewcommand{\choiceshook}{%
\settowidth{\leftmargin}{22.~~}%
}
\renewcommand{\partshook}{%
\settowidth{\leftmargin}{22.~~}%
}
%</cls>
%    \end{macrocode}
%
% 标号
%    \begin{macrocode}
%<*cls>
\renewcommand{\thepartno}{\arabic{partno}}
\renewcommand{\partlabel}{(\thepartno)}
%</cls>
%    \end{macrocode}
%
% 计分
%    \begin{macrocode}
%<*cls>
\pointname{分}
\pointformat{\kaishu (\thepoints)}
%</cls>
%    \end{macrocode}
%
% 填空题、选择题、简答题
%    \begin{macrocode}
%<*cfg>
\def\@tiankong@zongfen{56}
\def\@tiankong@tishu{14}
\def\@tiankong@fen{4}
\def\@xuanze@zongfen{16}
\def\@xuanze@tishu{4}
\def\@xuanze@fen{4}
\def\@jianda@zongfen{74}
\def\@jianda@tishu{5}
%</cfg>
%<*cls>
\newcounter{@dati}
\newcommand\shetk[3]{
  \def\@tiankong@zongfen{#1}
  \def\@tiankong@tishu{#2}
  \def\@tiankong@fen{#3}
}
\newcommand\tiankong{
  \stepcounter{@dati} 
  \fullwidth{\heiti \chinese{@dati}~~填空题(\kaishu 本大题满分~\@tiankong@zongfen~分)~\heiti 本大题有~\@tiankong@tishu~题,考生应在答题纸相应编号的空格内直接写结果,每个空格填对得~\@tiankong@fen~分,否则一律得零分.}
}
\newcommand\shexz[3]{
  \def\@xuanze@zongfen{#1}
  \def\@xuanze@tishu{#2}
  \def\@xuanze@fen{#3}
}
\newcommand\xuanze{
  \stepcounter{@dati} 
  \fullwidth{\heiti \chinese{@dati}.~~选择题(\kaishu 本大题满分~\@xuanze@zongfen~分)~\heiti 本大题共有~\@xuanze@tishu~题,每题有且只有一个正确答案,考生应在答题纸的相应编号上,将代表答案的小方格涂黑,选对得~\@xuanze@fen~分,否则一律得零分.}
}
\newcommand\shejd[2]{
  \def\@jianda@zongfen{#1}
  \def\@jianda@tishu{#2}
}
\newcommand\jianda{
  \qformat{\hskip\labelsep \kaishu \thequestion.~~(本题满分~\totalpoints~分)\hfill}
  \stepcounter{@dati}
  \fullwidth{\heiti \chinese{@dati}.~~简答题(\kaishu 本大题满分~\@jianda@zongfen~分)~\heiti 本大题共有~\@jianda@tishu~题,解答下列各题必须在答题纸相应的编号规定区域内写出必要的步骤.}
}
%</cls>
%    \end{macricode}
%
% solution环境
%    \begin{macrocode}
%<*cls>
\renewcommand{\solutiontitle}{\noindent\heiti{解:}\noindent}
%</cls>
%    \end{macrocode}

% 读取配置文件
%    \begin{macrocode}
%<*cls>
\AtEndOfPackage{\makeatletter%%
%% This is file `BHCexam.cfg',
%% generated with the docstrip utility.
%%
%% The original source files were:
%%
%% BHCexam.dtx  (with options: `cfg')
%% 
%% This is a generated file.
%% 
%% Copyright (C) 2014 by Charles Bao <charley792@gmail.com>
%% 
%% This file was generated from file(s) of the BHCexam package project.
%% ----------------------------------------------------------------------
%% 
%% It may be distributed and/or modified under the
%% conditions of the LaTeX Project Public License, either version 1.2
%% of this license or (at your option) any later version.
%% The latest version of this license is in
%%    http://www.latex-project.org/lppl.txt
%% and version 1.2 or later is part of all distributions of LaTeX
%% version 1999/12/01 or later.
%% 
%% This is the configuration file of the BHCexam package with LaTeX2e.
%% 
\ProvidesFile{BHCexam.cfg}
  [2016/03/01 v0.5 BHCexam
   configuration file]



\def\@biaoti{2011年全国普通高等学校招生统一考试(上海卷)}
\def\@fubiaoti{数学(理科)}
\def\@zongfen{150}
\def\@shijian{120}
\def\@tiankong@zongfen{56}
\def\@tiankong@tishu{14}
\def\@tiankong@fen{4}
\def\@xuanze@zongfen{16}
\def\@xuanze@tishu{4}
\def\@xuanze@fen{4}
\def\@jianda@zongfen{78}
\def\@jianda@tishu{5}
\endinput
%%
%% End of file `BHCexam.cfg'.
\makeatother}
%</cls>
%    \end{macrocode}
%
%

% \Finale
%
% \setcounter{IndexColumns}{2}
% \IndexPrologue{\section*{索引} {\it 意大利体的数字表示描述对应索引项的页码;
%                带下划线的数字表示定义对应索引项的代码行号;
%                罗马字体的数字表示使用对应索引项的代码行号。}}
%
% \GlossaryPrologue{\section*{版本更新}}
%
% \PrintIndex \PrintChanges
\endinput
